\documentclass[thesis.tex]{subfiles}
\begin{document}
\chapter{Background}

\section{The mobile ecosystem}\label{sec:mobile-ecosystem}
\todo{Describe the mobile ecosystem, and the requirements for a policy language
  to be able to talk about it.  This is basically related work, but only the
  \emph{really} pertinent stuff.}

This dissertation talks about the policies surrounding \emph{mobile
ecosystems}; but what, precisely, do we mean by the \emph{mobile
ecosystem}?  A cow after all is both mobile and has an ecosystem of
grass, farmers, and other cows surrounding it.  Succinctly when we
refer to the mobile ecosystem we are referring to \textbf{the
interactions surrounding the use of smart phones and tablet
computers}.  To fully understand the scale of the ecosystem however,
it is best to describe some of the history surrounding it.

\subsection{Mobile devices: the story so far}

Some of the earliest mobile devices were the \acp{PDA} devices of the
early 90s.  The earliest devices, such as Apple's Newton, were
portable miniature computers designed to store personal information,
calendars and notes.  Developers could even program additional apps
for them (in the case of the Newton using Apple's Dylan language).  At
the time mobile phones were just portable telephones, but starting
with the IBM Simon in 1994 these devices started to have some of the
functionality of the \ac{PDA}, becoming what would later be called a
\emph{smart phone}.  The big advantage of these early smart phones
over the \ac{PDA} was they had a telephone connection.  The earliest
devices allowed their users to access email and fax, as well as
managing personal information.  The early smart phones were somewhat
underpowered compared to the \ac{PDA} devices so both continued to
develop alongside one another, and started to become more affordable.

In 1998 the Symbian OS was released.  Developed by Psion (who made
\acp{PDA}), and Motorola, Ericsson and Nokia (all phone
manufacturers).  By the early 2000s it would start to become the
dominant smart phone OS.  These devices could install apps, written in
C++ or Java (if the phone supported JME). They had cameras, could play
music. They even had early malware which would illicitly send texts to
premium rate numbers.  They were the forebears of the \emph{modern}
smart phone.
They were also starting to become affordable, with the lowest end
models starting to being affordable by children and
teenagers\footnote{If you saved up for what seemed like
  forever$\ldots$}.  Devices like Nokia's N-Gage were marketed directly
towards these younger users and featured games users could buy for
their devices.

At this point in the mid-2000s we first start to see the mobile
ecosystem proper.  We have users with different devices, downloading
apps from different sources (not all legitimate).  These devices
contained personal information of the \acp{PDA}, but also photographs
and music.  User's could browse the web, and send each other pictures.
With increasing amounts of personal information malware authors
started to take note.  In a blog post from 2006 for Symantec, Chien
noted~\cite{eric_chien_spyware_2006}:

\begin{quote}
  ``While threats exist and are actively spreading, we are probably
  still years away from the situation we have with the Microsoft Windows
  [$\cdots$] We have already seen spyware applications for mobile devices
  (e.g. Spyware.Flexispy) that can monitor activities on the mobile
  device and then send them to a remote server. [$\cdots$] Just as
  worrying is the fact that the adware market is just beginning to take
  notice of mobile devices. Already some Bluetooth advertising schemes
  have been tested, where a bus stop is outfitted with a device that
  just spams out messages via Bluetooth.''
\end{quote}

Chien concluded his article saying:

\begin{quote}
  ``So, while worms and Trojans already exist for the mobile
  platforms, spyware and adware applications are just now gaining a
  foothold in the mobile device space. Spyware and adware pose a
  potentially large security issue in the near future, as the companies
  that produce such applications are less affected by the natural
  limiting factors.''
\end{quote}

This would turn out to be extremely prescient.


In 2008 Apple released the first iPhone, and shortly after Google
released the Android OS.  Symbian would release its final version in
2012, with Android becoming the dominant OS on most manufacturer's
devices.

\todo{Go on to talk about how users became increasingly aware of personal information being taken.}


\section{SecPAL}

SecPAL was originally intended to model and enforce access control
policies in grid computing
systems~\cite{becker_secpal:_2010}. Flexibility is part of SecPAL's
goals, however. At its core, SecPAL is a language with a simple
grammar (\autoref{fig:secpal-grammar}) and three evaluation rules
(\autoref{fig:secpal-rules}). The language's simplicity makes it easy
to apply to a new domain by instantiating it with predicates and
constraints that describe the domain. This simplicity does not come at
the cost of its expressiveness. SecPAL supports delegation, and
arbitrary constraints, as well as role and attribute based policies
styles. Other domains have successfully used variants of SecPAL to
describe their policies. Humphrey~\etal{} instantiated SecPAL with
predicates for the GridFTP protocol to create a Grid access control
policy language~\cite{humphrey_fine-grained_2007}. Aziz~\etal{}
created SecPAL4DSA by adding predicates for data-sharing
agreements~\cite{aziz_secpal4dsa:_2011}.  Becker~\etal{} added
predicates for describing \ac{PII}-handling preferences and created
SecPAL4P~\cite{becker_framework_2009}.

\begin{figure}
  \newcommand{\bracetext}[1]{\text{\sffamily #1}}
  \newcommand{\smalltext}[1]{\text{\ttfamily\small #1}}
  \centering
    \begin{equation*}
      \begin{array}{r l}
        \overbrace{\smalltext{`user'}}^{\bracetext{speaker}} &
        \smalltext{ says }\overbrace{\overbrace{\smalltext{ App }}^{\bracetext{subject}}\overbrace{\smalltext{ isRunnable}}^{\bracetext{predicate}}}^{\bracetext{fact}} \\
        & \overbrace{\smalltext{ if App isFree}}^{\bracetext{condition}} \\
        & \overbrace{\smalltext{ where hasPermission(App, `INTERNET') = true}}^{\bracetext{constraint}}.
      \end{array}
    \end{equation*}
  \caption{Structure of a SecPAL assertion.}
  \label{fig:assertion}
\end{figure}

\begin{figure}\centering
  \begin{tabular}{l p{0.7\linewidth}}
    \toprule
    $AC,\theta \vdash q$                     & Defining relation. A query assertion $q$ is valid given the assertions contained in the assertion context $AC$ and a variable substitution $\theta$. \\
    $\epsilon$                               & The empty substitution.                                                                                                                              \\
    \midrule
    $AC,\theta \vdash e \text{ says } fact$  & if $AC,\infty \models e\theta \text{ says } fact\theta$ and $dom(\theta) \subseteq vars(e \text{ says } fact)$                                       \\
    $AC,\theta_1\theta_2 \vdash q_1, q_2$    & if $AC,\theta_1 \vdash q_1$ and $AC,\theta_2 \vdash_2 q_2\theta_1$                                                                                   \\
    $AC,\theta \vdash q_1 \text{ or } q_2$   & if $AC,\theta \vdash q_1$ or $AC,\theta \vdash q_2$                                                                                                  \\
    $AC,\epsilon \vdash \mathsf{not}(q)$     & if $AC,\epsilon \not\vdash q$ and $vars(q) = \emptyset$                                                                                              \\
    $AC,\epsilon \vdash c$                   & if $\models c$                                                                                                                                       \\
    \bottomrule                             \\
  \end{tabular}
  \caption[SecPAL's semantics.]{SecPAL's semantics as described by Becker~\cite{becker_secpal:_2010}.}
\end{figure}

\newcommand{\bnfcomment}[1]{\slshape{\color{gray} (#1)}}
\newcommand{\secpal}[1]{\texttt{#1}}
\begin{figure}\footnotesize
  \begin{tabular}{r r l c}
    e          & $\Coloneqq$ & \secpal{x}                                       & \bnfcomment{variables}         \\
               & $\vert$     & \secpal{A}                                       & \bnfcomment{constants}         \\
    pred       & $\Coloneqq$ & \secpal{has} $\vert$ \secpal{can} $\vert$ \dots  & \bnfcomment{predicates}        \\
    D          & $\Coloneqq$ & 0                                                & \bnfcomment{no delegation}     \\
               & $\vert$     & $\infty$                                         & \bnfcomment{delegation}        \\
    vp         & $\Coloneqq$ & pred e$_1$ \dots e$_n$                           & \bnfcomment{verb phrase}       \\
               & $\vert$     & \secpal{can-say}$_D$ fact                       \\
               & $\vert$     & \secpal{can-act-as}  e                          \\
    f          & $\Coloneqq$ & e vp                                             & \bnfcomment{fact}              \\
    claim      & $\Coloneqq$ & f \secpal{if} f$_1$,\dots, f$_n$; c             \\
    assert     & $\Coloneqq$ & e \secpal{says} claim.                          \\
    AC         & $\Coloneqq$ & assert$_1$ \dots assert$_n$                      & \bnfcomment{assertion context} \\
    c          & $\Coloneqq$ & $\top$                                           & \bnfcomment{no constraint}     \\
               & $\vert$     & e$^\prime_1 =$ e$^\prime_2$                      & \bnfcomment{constraints}       \\
               & $\vert$     & \dots                                           \\
    e$^\prime$ & $\Coloneqq$ & e $\vert$ function(e$_1$,\dots e$_n$)           \\
  \end{tabular}
  \caption{BNF description of SecPAL.}
\label{fig:secpal-grammar}
\end{figure}
\end{document}

%%% Local Variables:
%%% mode: latex
%%% TeX-master: "../../thesis"
%%% End:
