\documentclass[thesis.tex]{subfiles}
\begin{document}
\chapter{Background}

\section{The mobile ecosystem}\label{sec:mobile-ecosystem}
\todo{Describe the mobile ecosystem, and the requirements for a policy language
  to be able to talk about it.  This is basically related work, but only the
  \emph{really} pertinent stuff.}

% \subsection{App Stores}\label{app-stores}
% 
% The Play Store is Google's marketplace for apps. It contains a large
% number of apps (around 2.2 million) uploaded by developers for users to
% buy. To use the market places user's can login with a Google account
% which is then used to sync purchases to their devices. Developers pay a
% small fee (currently \$25) to submit an app to the store.
% 
% Google publish guidelines for developers wanting to publish
% apps\footnote{\url{https://play.google.com/about/developer-content-policy/}}.
% These guidelines describe what content is acceptable, how developers may
% publicise their app and request feedback as well as what kinds of
% advertising is acceptable. Apps are inspected by Google before
% publication; a somewhat secretive process that includes static analysis
% as well as dynamic analysis using their \emph{Bouncer} tool
% \cite{oberheide_dissecting_2012}.
% 
% Whilst the Play store may be the \emph{official} Android app marketplace
% it is not the only way to install apps. Apps can be sideloaded
% (installed) by asking the operating system to install the APK file (the
% file format of an Android app) directly. Alternatively a third-party
% store can be installed which will allow users to buy and sideload apps.
% 
% These third-party stores exist for a variety of reasons. Some such as
% \emph{Qihoo 360 Mobile Assistant} and \emph{Baidu App Store} exist to cater to
% the Chinese market, which has not traditionally had access to Google's services
% for political reasons. Others such as the \emph{Amazon Appstore}\footnote{The
% Amazon app store is also available to all devices, and aims to be similar to the
% Google Play store, but with a more stringent review process.}, \emph{Sony Xperia
% Apps} or \emph{Samsung Galaxy Apps} exist as curated app stores featuring apps
% highlighting a particular device's features. There are app stores for open
% source apps like \emph{F-Droid}. There are app stores for devices that can't (or
% don't want to) pass the CTS and CDD like \emph{Yandex} and \emph{Aptoide}.
% 
% Each store comes with its own terms and conditions, some of which are
% summarised in Table \url{tab:store-summary}. A user who does not wish to
% be given updates might wish to use the Amazon store as it does not force
% them to be installed. A developer who wishes to sell pornographic apps,
% might want to use Yandex's store as many others prohibit them.
% 
% The only way to distinguish between these app stores is by
% careful consideration of their advertising and terms and conditions.
% 
% \begin{table*}\footnotesize
%   \begin{tabulary}{\linewidth}{lLLLL}
%     \toprule
%     & Google Play & Amazon & Yandex & Aptoide\\
%     \midrule
%     User ID & Name address and billing details. & Amazon ID. & None for free
%     apps, payment details for paid. & Contact details. No verification but
%     agreement not to lie.\\
%     Client info taken & Installation data, device ID, browsing history,
%     cookies. Can opt out. & Device ID, network info, location, usage data. &
%     Device ID, SIM number, Device content, System data, browsing history. &
%     Transaction history. They may share it with developers.\\
%     Customer Payments & Google Wallet and others at Google's discretion. &
%     Amazon. & Approved processor by Yandex or store operator. & Approved
%     processor by Aptoide.\\
%     Who is paid? & Google Commerce. & Amazon. & Developer. & Store
%     owner.\\
%     Prices set by & Developer. & Amazon. & Developer (but Yandex may
%     restrict to set values). & Developer and store owner.\\
%     Refunds & Only for defective or removed content. A refund may be
%     requested for two hours after purchase. & No. & Up to 15 minutes after
%     purchase. No for IAP. & Up to 24 hours after purchase.\\
%     Age of use & At least 13. & Any age (with consent of guardian). No
%     alcohol related content below 21. & At least 14. & A legal age to form a
%     contract with Aptoide.\\
%     Update provision & You agree to receive updates. & By default. & Yes for
%     security and bug-fixes. & Yes agree to receive updates.\\
%     Moderation & No obligation (but they may). & Publisher obliged to
%     provide info which may be used to give ratings. Amazon will not check
%     these ratings are accurate. & No obligation (but they may). & No
%     obligation (but they may). Trusted app mark does not indicate
%     moderation.\\
%     Acceptable use & No use as part of a public performance, or for
%     dangerous activities where failure may lead to death. & & & No
%     modification, rental, distribution or derivative works. You may use the
%     software.\\
%     Store rights to app & Marketing and optimising Android. & Distribution,
%     evaluation, modification, advertising, and creating derivatives for
%     promotion. & Advertising. & Modification and re-selling.\\
%     Withdrawal from sale & Immediate. & 10 days, or 5 days if for copy-write
%     reasons. & 90 days. A copy will be retained. & You may.\\
%     Developer ID & Google account and billing details. & Amazon ID. & Email,
%     company name, tax-id. & Email (preferably a Google developer
%     one).\\
%     EULA & Default offered. & Only if it doesn't interfere with Amazon's
%     terms. & Must be provided. & Default offered\\
%     Content restrictions & No alternate stores, sexual, violence, IP
%     infringing, PII publishing, illegal, gambling, malware, self-modifying
%     or system modifying content. No unpredictable network use. & No
%     offensive, pornography, illegal, IP infringing or gambling content. & No
%     defects. No illegal, disruptive, sexual, IP infringing, PII stealing,
%     alternative stores, or open-source content. & No displaying or linking
%     to illegal, privacy interfering, violent, PII stealing, IP infringing
%     content. Nothing \emph{spammy} or with unpredictable network
%     use.\\
%     Payout rates & 70\% of the user's payment. & 70\% list price (minus card
%     fees). & 70\% net-revenue (minus card fees). & 75\% revenue share (minus
%     card fees).\\
%     \bottomrule
%   \end{tabulary}
%   \label{tab:store-tandcs}
%   \caption{Summary of conditions in different stores.}
% \end{table*}

\section{SecPAL}

SecPAL was originally intended to model and enforce access control policies
    in grid computing systems~\cite{becker_secpal:_2010}. Flexibility is part of
    SecPAL's goals, however. At its core, SecPAL is a language with a simple grammar
    (\autoref{fig:secpal-grammar}) and three evaluation rules
    (\autoref{fig:secpal-rules}). The language's simplicity makes it easy to apply
    to a new domain by instantiating it with predicates and constraints that
    describe the domain. This simplicity does not come at the cost of its
    expressiveness. SecPAL supports delegation, and arbitrary constraints, as well
    as role and attribute based policies styles. Other domains have successfully
    used variants of SecPAL to describe their policies. Humphrey~\etal{} instantiated
    SecPAL with predicates for the GridFTP protocol to create a Grid access control
    policy language~\cite{humphrey_fine-grained_2007}. Aziz~\etal{} created SecPAL4DSA
    by adding predicates for data-sharing agreements~\cite{aziz_secpal4dsa:_2011}.
    Becker~\etal{} added predicates for describing \ac{PII}-handling preferences and
    created SecPAL4P~\cite{becker_framework_2009}.

\begin{figure}
  \newcommand{\bracetext}[1]{\text{\sffamily #1}}
  \newcommand{\smalltext}[1]{\text{\ttfamily\small #1}}
  \centering
    \begin{equation*}\small
      \begin{array}{r l}\footnotesize
        \overbrace{\smalltext{`user'}}^{\bracetext{speaker}} &
        \smalltext{ says }\overbrace{\overbrace{\smalltext{ App }}^{\bracetext{subject}}\overbrace{\smalltext{ isRunnable}}^{\bracetext{predicate}}}^{\bracetext{fact}} \\
        & \overbrace{\smalltext{ if App isFree}}^{\bracetext{condition}} \\
        & \overbrace{\smalltext{ where hasPermission(App, `INTERNET') = true}}^{\bracetext{constraint}}.
      \end{array}
    \end{equation*}
  \caption{Structure of a SecPAL assertion.}
  \label{fig:assertion}
\end{figure}

\begin{figure}\centering
  \begin{tabular}{l p{0.7\linewidth}}
    \toprule
    $AC,\theta \vdash q$                     & Defining relation. A query assertion $q$ is valid given the assertions contained in the assertion context $AC$ and a variable substitution $\theta$. \\
    $\epsilon$                               & The empty substitution.                                                                                                                              \\
    \midrule
    $AC,\theta \vdash e \text{ says } fact$  & if $AC,\infty \models e\theta \text{ says } fact\theta$ and $dom(\theta) \subseteq vars(e \text{ says } fact)$                                       \\
    $AC,\theta_1\theta_2 \vdash q_1, q_2$    & if $AC,\theta_1 \vdash q_1$ and $AC,\theta_2 \vdash_2 q_2\theta_1$                                                                                   \\
    $AC,\theta \vdash q_1 \text{ or } q_2$   & if $AC,\theta \vdash q_1$ or $AC,\theta \vdash q_2$                                                                                                  \\
    $AC,\epsilon \vdash \mathsf{not}(q)$     & if $AC,\epsilon \not\vdash q$ and $vars(q) = \emptyset$                                                                                              \\
    $AC,\epsilon \vdash c$                   & if $\models c$                                                                                                                                       \\
    \bottomrule                             \\
  \end{tabular}
  \caption[SecPAL's semantics.]{SecPAL's semantics as described by Becker~\cite{becker_secpal:_2010}.}
\end{figure}

\newcommand{\bnfcomment}[1]{\slshape{\color{gray} (#1)}}
\newcommand{\secpal}[1]{\texttt{#1}}
\begin{figure}\footnotesize
  \begin{tabular}{r r l c}
    e          & $\Coloneqq$ & \secpal{x}                                       & \bnfcomment{variables}         \\
               & $\vert$     & \secpal{A}                                       & \bnfcomment{constants}         \\
    pred       & $\Coloneqq$ & \secpal{has} $\vert$ \secpal{can} $\vert$ \dots  & \bnfcomment{predicates}        \\
    D          & $\Coloneqq$ & 0                                                & \bnfcomment{no delegation}     \\
               & $\vert$     & $\infty$                                         & \bnfcomment{delegation}        \\
    vp         & $\Coloneqq$ & pred e$_1$ \dots e$_n$                           & \bnfcomment{verb phrase}       \\
               & $\vert$     & \secpal{can-say}$_D$ fact                       \\
               & $\vert$     & \secpal{can-act-as}  e                          \\
    f          & $\Coloneqq$ & e vp                                             & \bnfcomment{fact}              \\
    claim      & $\Coloneqq$ & f \secpal{if} f$_1$,\dots, f$_n$; c             \\
    assert     & $\Coloneqq$ & e \secpal{says} claim.                          \\
    AC         & $\Coloneqq$ & assert$_1$ \dots assert$_n$                      & \bnfcomment{assertion context} \\
    c          & $\Coloneqq$ & $\top$                                           & \bnfcomment{no constraint}     \\
               & $\vert$     & e$^\prime_1 =$ e$^\prime_2$                      & \bnfcomment{constraints}       \\
               & $\vert$     & \dots                                           \\
    e$^\prime$ & $\Coloneqq$ & e $\vert$ function(e$_1$,\dots e$_n$)           \\
  \end{tabular}
  \caption{BNF description of SecPAL.}
\label{fig:secpal-grammar}
\end{figure}
\end{document}

%%% Local Variables:
%%% mode: latex
%%% TeX-master: "../../thesis"
%%% End:
