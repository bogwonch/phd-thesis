\documentclass[thesis.tex]{subfiles}
\begin{document}
\chapter{Conclusion}

% (1) What were your research question/s? This thesis posed the question/s (a)… (b)

This thesis looked at the ways the informal policies surrounding the mobile
ecosystem, linking the specification of the policies to their implementation. We
aimed to capture the trust relationships and policies using a formal language.
Using the formal description of the policies, we used them to measure the extent
users followed app privacy preferences, look for common concerns and idioms in
BYOD policies, and check policies for problems.

% (2) What were your answers to the research questions and how did you arrive at
%     them? My research study was describe in one sentence the kind of research you
%     undertook eg mixed methods study of… I found in answer to research question (a)
%     … and research question (b)… Don’t write more than about three sentences
%     or four or five bullets points in answer to each research question.

To explore these policies we instantiated and implemented a formal language,
AppPAL, based on the SecPAL authorization language to describe the policies of
the mobile ecosystem. Our implementation ran on Android and the JVM and included
tools to check policies for common errors, and analyze their contents.

By taking app installation data from a large number of users, and privacy
preferences discovered by researchers talking to users~\cite{lin_modeling_2014},
we modeled the preferences using AppPAL policies and measured the extent users
were following these policies. We found an example of the \emph{privacy
paradox}: users behavior didn't match their stated preferences.

Among other policies, we looked at five natural language BYOD policies and
translated them into AppPAL. This allowed us to identify two idioms,
acknowledgment and delegation, that whilst a significant part of the policies
had not been looked at by existing research and \ac{MDM} tools. These particular
idioms were of particular interest as they described social and often unenforced
aspects of policies whereas prior work and tools had looked predominantly at
the technical aspects.

% (3) What do your findings have to say to the literatures? Write an answer in
%     no more than four or five sentences. Think about whether your findings
%     challenge, trouble, suggest something (say what), add to what we know about x
%     from (name category of literatures and key authors), support (what)… My work
%     contributes to the literatures on… by…

Our work contributes to the literature by showing how informal language policies
can be captured formally and used as the basis for precise comparisons and
analysis. Existing work has focussed on the implementation aspects, but has
ignored the question of \emph{when} different policies should be enforced.
AppPAL lets us link the specification of a policy to its implementation. This
lets our policies be independent of any particular tool: we describe when any
tools should be run in order to check a policy rather than the precise,
mechanistic, checks of any particular tool.

% (4) What are the implications of this new knowledge? Who needs to know what
%     you have to say? Why? How could this knowledge be of interest/use to them? (Go
%     back to the policy or practice problem or think of a policy or practice problem
%     to which this knowledge speaks, or think about the ways in which literatures are
%     currently used/spoken that might be changed by your addition. Be careful not to
%     over or underclaim here.) What might happen as a result of knowing this new
%     stuff, your contribution? These findings could be of interest to… benefit to…
%     worry… Just write a few bullets or sentences in answer.

Our work also highlights that in many policies \emph{who} has performed a
particular check is sometimes more important to the policy the \emph{what} check
they actually performed. Existing research has looked at ways of enforcing
policies mechanistically, but sometimes just trusting the subject to follow them
on their own is sufficient to satisfy the policy. Existing policy tools focus on
the checks; but other aspects are also interesting to look at and provide
mechanisms for.

% (5) What further research might now be done as a result of your work? Here you
%     need to think about your work opening up a research agenda, being a building
%     block for further work that you, or someone else, might do. Write a few bullets
%     or sentences here. As a result of my study, further research might well be
%     conducted on/in order to …

The work described here is not without limitations. There are many policies in
the mobile ecosystem that we didn't fully examine such as the differences
between OSs, the differences between store contents. it could be that the reason
existing research has not looked at the trust relationships in mobile device
policies is because they are not interesting to anyone and just act as pointers
to others to set their own low level policies. We would disagree: precisely
specifying the informal policies leads to a better understanding of its
implications and the trust relationships describe the relationships between
companies and individuals that are often hidden by policies. Understanding the
responsibilities between each is interesting as it lets us see who is
responsible for what, and can be used to guide who to blame when things go
wrong.

This thesis focussed on the policies of the mobile ecosystem, but in the last
few years \ac{IoT} has established itself as its own ecosystem. These IoT
devices are even more distributed than mobile devices and frequently delegate to
webservices to provide information and process data. Our work on AppPAL could be
applied to IoT policies to model and examine the trust relationships and better
understand the relationships within.

% (6) Optional question Are there any implications for your own research
%     practice? What did you learn about researching from this study? Write a sentence
%     or two or a couple of bullets only.

In his Turing award lecture \emph{on trusting trust} Ken Thompson asked a
question~\cite{ken_thompson_reflections_1984}:
\begin{quotation} 
  To what extent should one trust a statement that a program is free of Trojan
  horses? Perhaps it is more important to trust the people who wrote the
  software.
\end{quotation}

He concluded:

\begin{quotation}
  The moral is obvious. You can't trust code that you did not totally create
  yourself. (Especially code from companies that employ people like me.)
  No amount of source-level verification or scrutiny will protect you from using untrusted code.''
\end{quotation} 

A similar question could be asked of the policies surrounding the mobile
ecosystem. Thompson's conclusion is still true: there is no way to trust that
any app you download is safe or good. There is no way to verify that an employee
isn't circumventing a BYOD policy in an evil and clever manner. It is infeasible
for users to create all their own apps and OSs, and few companies can only run
with one employee. We can't trust our policies are followed but we can
understand the trust relationships within them. If we understand the trust
relationships we can then decide how closely we want to check any part of it for
\emph{people like Ken}.

\end{document}

%%% Local Variables:
%%% mode: latex
%%% TeX-master: "../thesis"
%%% End:
