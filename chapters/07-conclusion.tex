\documentclass[thesis.tex]{subfiles}
\begin{document}
\chapter{Conclusion}

This thesis looked at capturing the security policies of the
mobile ecosystem precisely. We have tried to capture the trust
relationships within the policies using the AppPAL policy language we
instantiated from SecPAL. This allowed us to make precise comparisons
between different policies, and study the way policies were used in
practice rigorously.

Our work contributes to the literature by showing how to capture the
security policies of the mobile ecosystem, and how to use
the formal policies as the basis for comparisons and reasoning. Whilst
AppPAL does not have any new semantic language features, our work is
novel as it shows the application of policy languages to a new domain.

One benefit of AppPAL is that it lets us separate policy specification
from implementation. We can describe the policies of the mobile
ecosystem at a higher level than other tools, delegating to the
low-level tools when we need and when we want them to do their
analysis. This makes our policies independent of any particular tool,
and lets us abstract the checking process away from the reasons the
policy requires we do the checks.

In some policies, \emph{who} has performed a check is more important
to the policy than \emph{what} check they actually did. Existing
research has looked at ways of enforcing policies
mechanistically. Sometimes, however, just trusting the subject to
follow them on their own, however, is enough.  Capturing these trust
relationships lets us see precisely how a policy is satisfied.

Our work is not without its limitations. There are many policies in
the mobile ecosystem that we didn't fully look at; these include
policies such as the differences between OSs and the differences between
store contents. We have shown that by using AppPAL we can gain a
better understanding of the policies, however. This includes the
policy's implications and the trust relationships between companies,
people and tools they use.

\hspace{1em}

\noindent In his Turing award lecture \emph{on trusting trust} Ken
Thompson asked a question~\cite{ken_thompson_reflections_1984}:
\begin{quotation}\itshape\noindent
  ``To what extent should one trust a statement that a program is free
  of Trojan horses? Perhaps it is more important to trust the people who
  wrote the software.''
\end{quotation}
He concluded:
\begin{quotation}\itshape\noindent
  ``The moral is obvious. You can't trust code that you did not
  totally create yourself. (Especially code from companies that employ
  people like me.)  No amount of source-level verification or scrutiny
  will protect you from using untrusted code.''
\end{quotation} 
The same question could be asked of the policies in the mobile
ecosystem. Thompson's conclusion is still true: there is no way to
trust that any app you download is safe or good. There is no way to
check employees are not circumventing a BYOD policy cleverly. It is
hard for users to create all their own apps and OSs. Few companies can
only run with only one employee. We can't trust people follow our
policies. But we can understand the trust relationships within
them. If we understand the trust relationships, we can decide what
risks we take, and with whom.

\end{document}

%%% Local Variables:
%%% mode: latex
%%% TeX-master: "../thesis"
%%% End:
