\documentclass[thesis.tex]{subfiles}
\begin{document}
\chapter{Conclusion}

This thesis looks at capturing informal security policies precisely. We have
focussed on the policies of the mobile ecosystem, though the work could apply to
other domains. We have tried to capture the trust relationships within the
policies. This allows us to make precise comparisons and study the policies
rigorously. We instantiated and implemented AppPAL from the SecPAL authorization
language. We use AppPAL to describe the policies of the mobile ecosystem.

By taking user app installation data, and privacy
preferences found by researchers talking to users~\cite{lin_modeling_2014},
we modeled the user's preferences using AppPAL. We showed an example of the
privacy paradox: that user's privacy preferences do not seem to influence which
apps they install in practice.

We examined BYOD policies and
translated them into AppPAL. We identified idioms of
acknowledgment and delegation that 
had not been looked at by existing research and \ac{MDM} tools. These particular
idioms were of particular interest as they described social and often unenforced
aspects of policies. Prior work has looked predominantly at
the technical aspects of policies.   Our work focuses on the more informal aspects.

Our work contributes to the literature by showing how to model informal language policies,
and use them as the basis for precise comparisons and reasoning.
AppPAL lets us separate policy specification from implementation. This makes
our policies independent of any particular tool. We can abstract the checking
process away from the reason the policy requires we do the checks.

In some policies, \emph{who} has performed a
check is more important to the policy than \emph{what} check
they actually did. Existing research has looked at ways of enforcing
policies mechanistically. Sometimes just trusting the subject to follow them
on their own, however, is enough to satisfy the policy. 

Our work is not without its limitations. There are many policies in the mobile
ecosystem that we didn't fully look at; such as the differences between OSs, the
differences between store contents. We have shown that by using AppPAL we can
gain a better understanding of the policies, however. This includes the policy's
implications and the trust relationships between companies, people and tools
they use.

\hspace{1em}

\noindent In his Turing award lecture \emph{on trusting trust} Ken Thompson asked a
question~\cite{ken_thompson_reflections_1984}:
\begin{quotation}\itshape\noindent
  ``To what extent should one trust a statement that a program is free of Trojan
  horses? Perhaps it is more important to trust the people who wrote the
  software.''
\end{quotation}
He concluded:
\begin{quotation}\itshape\noindent
  ``The moral is obvious. You can't trust code that you did not totally create
  yourself. (Especially code from companies that employ people like me.)
  No amount of source-level verification or scrutiny will protect you from using untrusted code.''
\end{quotation} 
The same question could be asked of the policies in the mobile
ecosystem. Thompson's conclusion is still true: there is no way to trust that
any app you download is safe or good. There is no way to check employees
are not circumventing a BYOD policy in a clever way. It is hard
for users to create all their own apps and OSs. Few companies can only run
with only one employee. We can't trust people follow our policies. But we can
understand the trust relationships within them. If we understand the trust
relationships, we can decide what risks we take, and with whom.

\end{document}

%%% Local Variables:
%%% mode: latex
%%% TeX-master: "../thesis"
%%% End:
