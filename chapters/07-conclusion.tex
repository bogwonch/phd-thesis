\documentclass[thesis.tex]{subfiles}
\begin{document}
\chapter{Conclusion}

This thesis looks at the analyzing the informal policies surrounding of the
mobile ecosystem with authorization logic. We have tried to capture the trust
relationships within the policies and by to better understand the way policies
are used with mobile devices. 

To explore the mobile ecosystem we instantiated and implemented 
AppPAL from the SecPAL authorization language to describe the policies of
the mobile ecosystem.

By taking app installation data from a large number of users, and privacy
preferences discovered by researchers talking to users~\cite{lin_modeling_2014},
we modeled the user's preferences using AppPAL and have shown an example of the
privacy paradox: that user's privacy preferences do not seem to influence which
apps they install in practice.

We examined BYOD policies and
translated them into AppPAL. We identified idioms,
acknowledgment and delegation, that 
had not been looked at by existing research and \ac{MDM} tools. These particular
idioms were of particular interest as they described social and often unenforced
aspects of policies whereas prior work and tools had looked predominantly at
the technical aspects.

Our work contributes to the literature by showing how informal language policies
can be modeled and used as the basis for precise comparisons and reasoning.
AppPAL lets us link the specification of a policy to its implementation allowing
our policies be independent of any particular tool, and abstract the checking
process away from the reason why the policy requires we perform the checks.

In some policies \emph{who} has performed a
particular check is more important to the policy the \emph{what} check
they actually performed. Existing research has looked at ways of enforcing
policies mechanistically, but sometimes just trusting the subject to follow them
on their own is sufficient to satisfy the policy. 

Our work is not without its limitations. There are many policies in the mobile
ecosystem that we didn't fully examine such as the differences between OSs, the
differences between store contents. We have shown that by using AppPAL we can
gain a better understanding of the policies implications and the trust
relationships between companies, individuals and tools they refer to.

\hspace{1em}

\noindent In his Turing award lecture \emph{on trusting trust} Ken Thompson asked a
question~\cite{ken_thompson_reflections_1984}:
\begin{quotation}\itshape\noindent
  ``To what extent should one trust a statement that a program is free of Trojan
  horses? Perhaps it is more important to trust the people who wrote the
  software.''
\end{quotation}
He concluded:
\begin{quotation}\itshape
  ``The moral is obvious. You can't trust code that you did not totally create
  yourself. (Especially code from companies that employ people like me.)
  No amount of source-level verification or scrutiny will protect you from using untrusted code.''
\end{quotation} 
A similar question could be asked of the policies surrounding the mobile
ecosystem. Thompson's conclusion is still true: there is no way to trust that
any app you download is safe or good. There is no way to verify that an employee
isn't circumventing a BYOD policy in an evil and clever manner. It is infeasible
for users to create all their own apps and OSs, and few companies can only run
with one employee. We can't trust our policies are followed but we can
understand the trust relationships within them. If we understand the trust
relationships we can decide what risks we are willing to take, and with whom.

\end{document}

%%% Local Variables:
%%% mode: latex
%%% TeX-master: "../thesis"
%%% End:
