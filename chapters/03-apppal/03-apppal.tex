\documentclass[thesis.tex]{subfiles}
\begin{document}
\chapter{AppPAL}

\subsection{The Design and Implementation of AppPAL}

AppPAL is an instantiation of Becker et al.'s SecPAL\cite{becker_secpal:_2010} with constraints (statements checkable using information external to the language such as the time of day or static analysis tools) and predicates that allow us to decide which apps to run or install.
The language allows us to reason about apps using statements from third parties. AppPAL allows us to enforce the policies on a device.
I can express trust relationships amongst these parties and use constraints to do additional checks, such as using security checks.
This lets us enforce more complex policies than existing tools such as Kirin which are limited to permissions checks. 
Policies can be enforced by the stores selling the apps, on the devices installing apps or by third-parties providing app vetting services.

Consider the following example:
  a user, Alice, may have rules she has to follow when using apps for work and her own policies when using apps at home in her private life.
Using AppPAL I can write policies for work and home, and decide which policy to enforce using a user's location, or the time of day:

\begin{lstlisting}
'alice' says App isRunnable
  if 'home-policy' isMetBy(App)
  where at('work') = false.
\end{lstlisting}
\begin{lstlisting}
'alice' says App isRunnable
  if 'work-policy' isMetBy(App)
  where beforeHourOfDay('17') = true.
\end{lstlisting}

I can delegate policy specification to third parties or roles, and assign principals to roles:

\begin{lstlisting}
'alice' says 'it-department' can-say 'work-policy' isMetBy(App).
'alice' says 'alice' can-act-as 'it-department'.
\end{lstlisting}

I can write policies specifying which permissions an app must or must not have by its app store categorization.
For example, it would be okay allowing a photography app access to the camera, but not to allow access to location data if the user doesn't want their photos geotagged.

\begin{lstlisting}
'alice' says App isRunnable
  if 'permissions-policy' isMetBy(App).
'alice' says 'permissions-policy' isMetBy(App)
  if App isAnApp
  where
    category(App, 'Photography'),
    hasPermission(App, 'LOCATION') = false,
    hasPermission(App, 'CAMERA') = true.
\end{lstlisting}

AppPAL is implemented as a library for 
The parser is implemented using ANTLR4.
AppPAL's basic syntax is inherited from SecPAL and (shown in \autoref{fig:assertion-grammar}).

\begin{figure}
  \newcommand{\bracetext}[1]{\text{\sffamily #1}}
  \newcommand{\smalltext}[1]{\text{\ttfamily\small #1}}
  \centering
    \begin{equation*}\small
      \begin{array}{r l}\footnotesize
        \overbrace{\smalltext{`user'}}^{\bracetext{speaker}} &
        \smalltext{ says }\overbrace{\overbrace{\smalltext{ App }}^{\bracetext{subject}}\overbrace{\smalltext{ isRunnable}}^{\bracetext{predicate}}}^{\bracetext{fact}} \\
        & \overbrace{\smalltext{ if App isFree}}^{\bracetext{condition}} \\
        & \overbrace{\smalltext{ where hasPermission(App, `INTERNET') = true}}^{\bracetext{constraint}}.
      \end{array}
    \end{equation*}
  \caption{Structure of an AppPAL assertion.}
  \label{fig:assertion-structure}
\end{figure}

\begin{figure}
  \newcommand{\nonterminal}[1]{$\langle$#1$\rangle$}
  \newcommand{\terminal}[1]{\textbf{#1}}
  \begin{tabular}{r c l}
    \footnotesize
    \nonterminal{Assertion} & $\coloneqq$ & \nonterminal{E} \terminal{says} \nonterminal{Fact} \\
                            &             & \hspace{1em}(\terminal{if} (\nonterminal{Fact}\terminal{,})+)?\terminal{.} \\
                            &             & \hspace{1em}(\terminal{where} \nonterminal{Constraint})? \\
    \nonterminal{Fact}      & $\coloneqq$ & \nonterminal{E} (\terminal{isRunable} $\vert$ $\ldots$) \\
                            & $\vert$     & \nonterminal{E} \terminal{can-say} \terminal{inf}? \nonterminal{Fact} \\
                            & $\vert$     & \nonterminal{E} \terminal{can-act-as} \nonterminal{E} \\
    \nonterminal{E}         & $\coloneqq$ & \terminal{Variable} $\vert$ \terminal{`constant'}
  \end{tabular}
  \caption{Grammar of an AppPAL assertion.}
  \label{fig:assertion-grammar}
\end{figure}

\begin{figure}
  \centering
  \begin{eqnarray*}
    \infer[\textsf{\scriptsize cond}]{%
      AC, D \models A\textsf{~says~}fact\theta
    }{%
      \begin{array}[c]{c}
        \left(A\textsf{~says~}\textit{fact}\textsf{~if~}\textit{fact}_1, \ldots, \textit{fact}_k, c\right) \in AC \\
        \forall i \in [1\cdots k]. AC,D\models A\textsf{~says~}\textit{fact}_i\theta
      \end{array}
      & \models{c\theta}
      & \textsf{vars}(\textit{fact}\theta) = \emptyset)
    }\\
    \infer[\textsf{\scriptsize can say}]{%
      AC, \infty \models A\textsf{~says~}\textit{fact}
    }{%
      AC, \infty \models A\textsf{~says~}B\textsf{~can~say}_D \textit{fact}
      & AC, D \models B\textsf{~says~}\textit{fact}
    } \\
    \infer[\textsf{\scriptsize can act as}]{%
      AC, D \models A\textsf{~says~}B~\textit{verbphrase}
    }{%
      AC, D \models A\textsf{~says~}B\textsf{~can~act~as~}C
      & AC, D \models A\textsf{~says~}C~\textit{verbphrase}
    }
  \end{eqnarray*}
  \caption[Inference rules used to evaluate {SecPAL}.]{The inference rules used to evaluate {SecPAL}. All {SecPAL} rules are
  evaluated in the context of a set of other assertions $AC$ as well as an
  allowed level of delegation $D$ which may be $0$ or $\infty$.}
\label{secpal:rules}
\end{figure}


\subsection{Why SecPAL?}

For modelling and writing policies I want to start from a policy
language that supported several key features; namely:

\begin{itemize}
  \item Model decisions at separate locations.  Mobile ecosystems are
    inherently distributed, and decisions made by one device may not be
    the same as the decisions made by another device.  They may, however
    wish to share information.  By opting for a distributed language we
    can check the policies locally and make a decision based on that
    user's information rather than by deferring to an all-seeing policy
    enforcer.  This suits the problem better because individual devices
    and stores have to make decisions on their own and may not have access
    to all known information.

  \item Calling external functions that can obtain extra information
    about entities that may require special actions to fetch: for example
    an explicit ``okay'' from the user, or some metadata about an app like
    its version number.  If I don't allow the ability to fetch this
    information on-the-fly then it will have to be requested before
    checking the policy. 

    I also want to be able to use static analysis tools as part of
    our policies.  These tools can infer complex properties of apps, but
    the low-level details they check for needn't be expressable in our
    high level policies.  For example it is not necessary to encode how
    data moves over intents in AppPAL policies, but it is necessary to be
    able to say some app leaks information to another app.  A tool, such
    as FlowDroid can check for this, and there is no need to replicate the
    checking in AppPAL if I can simply call out to FlowDroid to give us a
    decision.

  \item Delegation.  I want to make a decision based on information from an
    external source.  This suggest that a \emph{can-say} mechanism is required
    that will allow us to express how trust is distributed.   Since the
    relationship between app stores and devices is one where a user delegates
    trust to the store and the device uses the external store to obtain the app
    this would hint that being able to express delegation will allow us to write
    more accurate policies.
\end{itemize}

Additionally I also wanted a language that was readable and easy to extend.
SecPAL was chosen as it met our requirements.

\section{Changes from SecPAL}

\subsection{Typed SecPAL}
As part of SecPAL's safety condition all variables in an assertions head must be
used in the body.  When trying to describe policies we found a common pattern is
to give a person who satisfied some property (for example they were a staff
member) the ability to say statements about apps.
For example, Alice might be willing to let her friends say what apps are
suitable for her children.  This could be expressed in SecPAL as follows:
\begin{lstlisting}
'alice' says Friend can-say App isSuitableFor(Child)
  if Friend isFriend,
     App isApp,
     Child isChild.
\end{lstlisting}
The conditionals in this assertion add unnecessary noise to the assertion. We
know from the names of the variable what set of constants might be used to used
to instantiate it (Alice's friends, apps or children). To avoid this noise I
added a sugared syntax to AppPAL that allows variables to declare their
\emph{type}.  Using the sugared notation the above statement becomes:
\begin{lstlisting}
'alice' says Friend:F can-say App:A isSuitableFor(Child:C).
\end{lstlisting}
\begin{marginfigure}
  \newcommand{\nonterminal}[1]{$\langle$#1$\rangle$}
  \newcommand{\terminal}[1]{\textbf{#1}}
  \begin{tabular}{r c l}
    \footnotesize
    \nonterminal{E}         & $\coloneqq$ & \nonterminal{Variable} $\vert$ \terminal{'constant'} \\
    \nonterminal{Variable}  & $\coloneqq$ & \terminal{Type}\terminal{:}\terminal{VariableName} \\
                            & $\vert$     & \terminal{VariableName}
  \end{tabular}
  \caption{Changes to SecPAL's variable syntax.}
  \label{fig:apppal-types}
\end{marginfigure}
The changes to SecPAL's syntax is shown is \autoref{fig:apppal-types}.
After an assertion has been parsed the variables with types are extracted for
each variable a conditional is added that \texttt{VariableName \emph{is}Type},
and the types are removed from the variables.

This gives a cleaner policy language however it also means that the predicates
used start to have some intrinsic meaning.  If a predicate starts with
\texttt{is} then it is describing some property of the predicates subject.
SecPAL did not require predicates follow any naming conventions, however with
AppPAL we have started to give predicates meaning based on their name.

\subsection{Predicate Conventions}

Following on from the \texttt{is} predicate in the typing syntax, when writing
AppPAL policies I try and limit the predicates to four kinds:
\begin{description}
\item[\bfseries\texttt{subject \emph{is}Type}]
  A typing statement.  The \emph{subject} is an example of the \emph{type}.  An
  example might be that \lstinline!'anrgry-birds' isApp! or that
  \lstinline!'jennie' isEmployee!.
\item[\bfseries\texttt{subject \emph{has}Action}]
  A statement of action.  The \emph{subject} has carried out an \emph{action} in
  the past. For example if an app has requested a perimission then we write
  \lstinline!App:A hasRequestedPermission(Permission:P)!, if a device requires
  its owner to grant a permission we might write
\begin{lstlisting}
'device' says User:U can-say 
  App:A hasBeenGranted(Permission:P)
  if 'device' isOwnedBy(U).
\end{lstlisting}
\item[\bfseries\texttt{subject \emph{can}Action}]
  An authorization. The \emph{subject} is permitted to carry out the \emph{action}.
  For example \lstinline!Device:D canInstall(App:A)! to say what apps a device
  can install or \lstinline!App:A canConnectTo(URL:U)! to describe a limitation
  on app network abilities.
\item[\bfseries\texttt{subject \emph{must}Action}]
  An obligation.  The \emph{subject} should carry out the \emph{action} as soon
  as possible.
  An example might be requiring the device inform a company's IT department if
  there have been three unsuccessful password attempts:
\begin{lstlisting}
'company' says Device:D mustInform('it', 'login-failure')
  if D hasUnsuccesfulLogins(N)
  where N >= 3.
\end{lstlisting}
\end{description}

\section{Examples of AppPAL}

Consider the following example: a user, Alice, may have rules she has to follow
when using apps for work and her own policies when using apps at home in her
private life. Using AppPAL I can write policies for work and home, and decide
which policy to enforce using a user's location, or the time of day:

\begin{lstlisting}
'alice' says App isRunnable
  if 'home-policy' isMetBy(App)
  where at('work') = false.
\end{lstlisting}
\begin{lstlisting}
'alice' says App isRunnable
  if 'work-policy' isMetBy(App)
  where beforeHourOfDay('17') = true.
\end{lstlisting}

I can delegate policy specification to third parties or roles, and assign principals to roles:

\begin{lstlisting}
'alice' says 'it-department' can-say 'work-policy' isMetBy(App).
'alice' says 'alice' can-act-as 'it-department'.
\end{lstlisting}

I can write policies specifying which permissions an app must or must not have
by its app store categorization. For example, it would be okay allowing a
photography app access to the camera, but not to allow access to location data
if the user doesn't want their photos geotagged.

\begin{lstlisting}
'alice' says App isRunnable
  if 'permissions-policy' isMetBy(App).
'alice' says 'permissions-policy' isMetBy(App)
  if App isAnApp
  where
    category(App, 'Photography'),
    hasPermission(App, 'LOCATION') = false,
    hasPermission(App, 'CAMERA') = true.
\end{lstlisting}

\section{Implementation}

AppPAL is implemented as a library for Android and Java.
The parser is implemented using ANTLR4.
AppPAL's syntax is inherited from SecPAL~\cite{becker_secpal:_2010} (shown in \autoref{fig:assertion}).


\end{document}

%%% Local Variables:
%%% mode: latex
%%% TeX-master: "../../thesis"
%%% End:
