
\chapter{Probabilistic SecPAL Changes and Evaluation}
\label{appendix:probabilistic}

In \autoref{chap:future-work} we described how SecPAL could be modified so that
assertions could carry a measure of how plausible the speaker believed them to
be. To implement this we would require changes to SecPAL's evaluation algorithm.
Also when describing SecPAL Becker showed that it could be translated into
Datalog\textsuperscript{C} in order to show that the language was tractable
could be evaluated in polynomial time~\cite{becker_secpal:_2010}.
Becker~\etal~gave in their technical report an algorithm (5.2) translating
SecPAL into Datalog\textsuperscript{C}. By modifying this algorithm to add the
plausibility annotations described in this chapter, we can show the Plausible
SecPAL is also tractable.

 The remainder of this appendix presents the \new{modifications} to SecPAL's
evaluation rules, as well as Becker~\etal{}'s Algorithm~5.2 \textsf{as described
by Becker \new{with additions for plausibility}}. We do not present any
arguments for the correctness of our modifications, but rather present them as a
tentative first step towards implementing Plausible SecPAL.

\section{Evaluating Plausibility}

SecPAL has three rules for evaluation: \emph{cond}, \emph{can-say},
and \emph{can-act-as}.  We modify the language so that the \emph{says}
keyword has an an annotation $0 \geq p \geq 1$ denoting a statements
\emph{plausibility}.  If the annotation is missing then it is assumed
to be $1$ We also assume a plausibility combining function $\oplus$
which combines plausibility, though we do not define it.

{\footnotesize\centering
\begin{equation*}
  \infer[\text{cond\new{$\leq$}}]{
    \AC{}, D \models A~\says{\bigoplus_{i=1}^n p_i}~f\theta
  }{
    \begin{matrix}{
      \left(A~\says{}~f~if~f_1\cdots f_n~\where~c~\new{\text{with plausibility at least}~p_{lim}}\right) \in \AC{}
    }\\{
      \forall i \in [1\cdots n]. \AC{}, D \models A~\says{p_i}~f_i\theta
    }\\\new{
      0 < p_{lim} \leq \bigoplus_{i=1}^n p_i
    }
    \end{matrix}&
    \vdash c\theta &
    vars\left(f\theta\right) = \emptyset
  }
\end{equation*}
\begin{equation*}
  \infer[\text{cond\new{=}}]{
    \AC{}, D \models A~\says{p_{lim}}~f\theta
  }{
    \begin{matrix}{
      \left(A~\says{}~f~if~f_1\cdots f_n~\where~c~\new{\text{with plausibility is}~p_{lim}}\right) \in \AC{}
    }\\{
      \forall i \in [1\cdots n]. \AC{}, D \models A~\says{p_i}~f_i\theta
    }\\\new{
      0 < p_{lim} \leq \bigoplus_{i=1}^n p_i
    }
    \end{matrix}&
    \vdash c\theta &
    vars\left(f\theta\right) = \emptyset
  }
\end{equation*}
\begin{equation*}
  \infer[\text{can-say}]{
    \AC{}, \infty \models A~\says{p_1 \oplus p_2}~f
  }{
    \AC{}, \infty \models A~\says{p_1}~B~\canSay{D}~f &
    \AC{}, D \models B~\says{p_2}~f
  }
\end{equation*}
\begin{equation*}
  \infer[\text{can-act-as}]{
    \AC{}, D \models A~\says{p_1 \oplus p_2}~x~vp
  }{
    \AC{}, D \models A~\says{p_1}~x~\canActAs~y &
    \AC{}, D \models B~\says{p_2}~y~vp
  }
\end{equation*}
\begin{equation*}
  \new{
    \infer[\text{reduce}]{
        \AC{}, D \models A~\says{p}~f
    }{
        \AC{}, D \models A~\says{p^\prime}~f & p \leq p^\prime
    }
  }
\end{equation*}
}

Any derived statement is at most as plausible as the
combination of the statements that went into deriving it.  

We split the \emph{cond} rule into two variants. The \emph{cond$\leq$} rule
allows us to specify a minimum plausibility required by combining all the
conditional statements and if that limit is exceeded we take the combined
probability in the plausibility of the outcome. The \emph{cond=} rule allows us
again to set a minimum plausibility but this time we take the stated
plausibility if the rule is satisfied. These two \emph{cond} rules serve
different purposes. The \emph{cond=} variant is useful when we want to set a
limit on the plausibility: for instance when we have a tool with a known
confidence rate we want to run, or a fact which we know how plausible it is. The
\emph{cond=} rule is useful for when you want to ensure that a decision is made
with a certain least-confidence, for instance if you want to be at least 80\%
sure that an app is safe to use before doing anything with it. In this case we
would want the combined plausibility to trickle through the proof not the lower
limit.

We also add a plausibility reduction rule that allows us to reduce the
plausibility of an assertion, this allows us to phrase a policy query
as \emph{``is it at least 50\% plausible that...''} rather than having
to discover the plausibilities precisely.

\section{A Plausible Algorithm 5.2}

\begin{quotation}
  \sffamily
  
We now describe an algorithm for translating an assertion context into
an equivalent constrained Datalog program. We treat expressions of the
form $e_1 says_k fact$ as Datalog literals, where $k$ is either a
variable or 0 or $\infty$. This can be seen as a sugared notation for
a literal where the predicate name is the string concatenation of all
infix operators (\textsf{says}, \textsf{can-say}, \textsf{can-act-as},
and predicates) occurring in the expression, including subscripts for
\textsf{can-say}. The arguments of the literal are the collected
expressions between these infix operators. For example, the expression
$$A~says_k^{\new{p}}~x~can say_\infty~y~can say_0~B~can act as~z$$
is shorthand for:
\begin{center}
\texttt{says\_cansay\_infinity\_cansay\_zero\_canactas(A,\new{p},k,x,y,B,z)}.
\end{center}

Given an assertion: 

\begin{center} \lstinline!$A$ says $f_0$ if $f_1\cdots f_n$ where $c$ with plausibility $p$.! \end{center}

\begin{enumerate}
\item 
  If $f_0$ is flat (it isn't a can-say statement), then the assertion is translated into the clause:
  \begin{lstlisting}[language=Prolog]
$A$ says$_k^{\new{p_*}}$ $f_0$ :- 
    $A$ says$_k^{\new{p_1}}$ $f_1$ $\cdots$ $A$ says$_k^{\new{p_n}}$ $f_n$, c, 
    $\new{p_\Sigma \text{ is } p_1 \oplus \cdots \oplus p_n}$, 
    $\new{0 < p_{lim} \leq p_\Sigma}$.
  \end{lstlisting}
  Where $k$ is a fresh variable \new{and $p_*$ is $p_{lim}$ if the plausibility is \texttt{is}, and $p_\Sigma$ is it is \texttt{at least}}.
  
\item 
  Otherwise $f_0$ is of the form:

  \lstinline!$e_0$ can-say $D_0$ $\cdots$ $e_{n-1}$ can-say $D_{n-1}$ $f$!

  Where $f$ is flat. Let:

  $f^\prime_n \equiv f$ and \lstinline!$f^\prime_i \equiv e_i$ can-say $D_i$ $f^\prime_{i+1}$!, for $i\in\left\{0\cdots n-1\right\}$.

  Note that $f_0 = f^\prime_0$.  

  Then the assertion \lstinline!$A$ says $f_0$ if $f_1\cdots f_m$, c, with plausibility $p$! is translated into a set of $n+1$ Datalog rules as follows.
  
  \begin{enumerate}
  \item 
    We add the Datalog rule:
    \begin{lstlisting}[language=Prolog]
$A$ says$_k^{\new{p_*}}$ $f^\prime_0$ :-
    $x$ says$_k^{\new{p_1}}$ $f_1\cdots$ $A$ says$_k^{\new{p_m}}$ $f_m$, c,
    $\new{p_\Sigma \text{ is } p_1 \oplus \cdots \oplus p_n}$, 
    $\new{0 < p_{lim} \leq p_\Sigma}$.
    \end{lstlisting}
    Where $k$ is a fresh variable, \new{and $p_*$ is $p_{lim}$ if the plausibility is \texttt{is}, and $p_\Sigma$ is it is \texttt{at least}}.

  \item
    For each $i\in\left\{1\cdots n\right\}$, we add a Datalog rule
    \begin{lstlisting}[language=Prolog]
$A$ says$_\infty^{\new{p_*}}$ $f^\prime_i$ :-
    $x$ says$_{D_{i-1}}^{\new{p_1}}$ $f^\prime_i$,
    $A$ says$_{\infty}^{\new{p_2}}$ $x$ can-say $D_{i-1}$ $f^\prime_i$,
    $\new{p_* \text{ is } p_1\oplus p_2}$, 
    $\new{0 < p_* \leq 1}$.
    \end{lstlisting}
    Where $x$ is a fresh variable.
  \end{enumerate}
  
  \item
    For each Datalog rule created above of the form:
    \begin{lstlisting}[language=Prolog]
      $A$ says$_k^p$ $e$ $v$ :- $\cdots$
    \end{lstlisting}
    we add a rule:

    \begin{lstlisting}[language=Prolog]
$A$ says$_\infty^{\new{p_*}}$ $e$ $v$ :-
    $x$ says$_{k}^{\new{p_1}}$ $x$ can-act-as $e$,
    $A$ says$_{k}^{\new{p_2}}$ $e$ $v$,
    $\new{p_* \text{ is } p_1\oplus p_2}$, 
    $\new{0 < p_* \leq 1}$.
    \end{lstlisting} Where $x$ is a fresh variable.  Note that $k$ is
not a fresh variable, but either a constant or a variable taken from
the original rule.
    

    \new{
      We also add an additional rule (to account for the reduce rule)
      that should not be used in general, but only when trying to reduce the
      plausibility to account for a lower bound on plausibility in a query:
      }
      
      \begin{lstlisting}[basicstyle=\color{BrickRed}\ttfamily]
$A$ says$_k^{p_\downarrow}$ $e$ $v$ :- 
    $A$ says$_k^p$ $e$ $v$,
    $p_\downarrow \leq p$.
      \end{lstlisting}
\end{enumerate}
\end{quotation}
